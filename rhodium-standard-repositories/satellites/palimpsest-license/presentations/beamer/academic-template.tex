% Palimpsest License: Academic Presentation Template (LaTeX Beamer)
% British English, academic rigour, citation-ready

\documentclass[aspectratio=169,11pt]{beamer}

% Packages
\usepackage[british]{babel}
\usepackage[utf8]{inputenc}
\usepackage[T1]{fontenc}
\usepackage{lmodern}
\usepackage{amsmath,amssymb,amsthm}
\usepackage{graphicx}
\usepackage{listings}
\usepackage{xcolor}
\usepackage{hyperref}
\usepackage{biblatex}
\usepackage{csquotes}

% Bibliography
\addbibresource{references.bib}

% Theme customisation
\usetheme{Madrid}
\usecolortheme{seahorse}

% Custom colours (adjust for your institution)
\definecolor{palimprimary}{RGB}{139,71,137}
\definecolor{palimpsecondary}{RGB}{74,124,126}
\definecolor{palimaccent}{RGB}{212,175,55}

\setbeamercolor{structure}{fg=palimprimary}
\setbeamercolor{frametitle}{bg=palimprimary,fg=white}
\setbeamercolor{block title}{bg=palimpsecondary,fg=white}
\setbeamercolor{block body}{bg=palimpsecondary!10,fg=black}

% Code listings
\lstset{
    basicstyle=\ttfamily\small,
    keywordstyle=\color{palimprimary}\bfseries,
    commentstyle=\color{gray}\itshape,
    stringstyle=\color{palimaccent},
    showstringspaces=false,
    breaklines=true,
    frame=single,
    numbers=left,
    numberstyle=\tiny\color{gray}
}

% Title information
\title{The Palimpsest License}
\subtitle{Legal and Technical Framework for Protecting Creative Work in the AI Age}
\author{Your Name\inst{1} \and Co-Author Name\inst{2}}
\institute[Universities]{
    \inst{1}Department of Law, University of Example\\
    \inst{2}School of Computer Science, University of Example
}
\date{Conference Name, November 2025}

% Footer customisation
\setbeamertemplate{footline}{
    \leavevmode%
    \hbox{%
        \begin{beamercolorbox}[wd=.333333\paperwidth,ht=2.25ex,dp=1ex,center]{author in head/foot}%
            \usebeamerfont{author in head/foot}\insertshortauthor
        \end{beamercolorbox}%
        \begin{beamercolorbox}[wd=.333333\paperwidth,ht=2.25ex,dp=1ex,center]{title in head/foot}%
            \usebeamerfont{title in head/foot}\insertshorttitle
        \end{beamercolorbox}%
        \begin{beamercolorbox}[wd=.333333\paperwidth,ht=2.25ex,dp=1ex,right]{date in head/foot}%
            \usebeamerfont{date in head/foot}\insertshortdate{}\hspace*{2em}
            \insertframenumber{} / \inserttotalframenumber\hspace*{2ex}
        \end{beamercolorbox}%
    }%
    \vskip0pt%
}

\begin{document}

% Title slide
\frame{\titlepage}

% Table of contents
\begin{frame}{Outline}
    \tableofcontents
\end{frame}

% ===== SECTION 1: INTRODUCTION =====
\section{Introduction}

\begin{frame}{Research Context}
    \begin{block}{Motivating Problem}
        Traditional copyright law, codified in the Berne Convention (1886) and WIPO Copyright Treaty (1996), fails to adequately address artificial intelligence's use of creative works as training data.
    \end{block}

    \vspace{0.5cm}

    \begin{itemize}
        \item AI systems scrape millions of creative works without consent
        \item Current legal frameworks offer limited recourse
        \item Attribution and cultural context systematically erased
        \item Quantum computing threatens existing cryptographic attribution
    \end{itemize}

    \vspace{0.5cm}

    \textbf{Research Question:} Can a novel licensing framework provide enforceable protections for creative work in the AI age whilst enabling ethical AI development?
\end{frame}

\begin{frame}{Literature Review}
    \begin{table}[h]
        \centering
        \small
        \begin{tabular}{|l|l|l|}
            \hline
            \textbf{Domain} & \textbf{Gap} & \textbf{Our Contribution} \\
            \hline
            IP Law & No AI-specific consent frameworks & Clause 1.2 consent template \\
            \hline
            Technical & Attribution breaks with AI & Quantum-resistant SLTs \\
            \hline
            Cultural & No collective rights support & DAO governance (Clause 3.6) \\
            \hline
            Ethics & Emotional context ignored & Emotional lineage protection \\
            \hline
        \end{tabular}
    \end{table}

    \vspace{0.3cm}

    \textbf{Key Citations:}
    \begin{itemize}
        \item \cite{samuelson2023} — Fair use inadequate for AI training
        \item \cite{levendowski2018} — Cultural erasure in AI outputs
        \item \cite{henderson2023} — DAO legal personality challenges
    \end{itemize}
\end{frame}

% ===== SECTION 2: METHODOLOGY =====
\section{Methodology}

\begin{frame}{Research Design}
    \textbf{Mixed-Methods Approach:}

    \begin{enumerate}
        \item \textbf{Legal Analysis}
            \begin{itemize}
                \item Comparative jurisdiction study (Dutch, Scottish, EU, US)
                \item Doctrine analysis (moral rights, contract law)
                \item Case law review (precedent identification)
            \end{itemize}

        \item \textbf{Technical Development}
            \begin{itemize}
                \item Cryptographic implementation (CRYSTALS-Dilithium)
                \item Protocol design (AIBDP specification)
                \item Metadata standards (JSON-LD, RDF integration)
            \end{itemize}

        \item \textbf{Empirical Validation}
            \begin{itemize}
                \item Case study: Diaspora poetry platform (N=50,000 works)
                \item Performance benchmarking
                \item Creator interviews (N=30)
            \end{itemize}
    \end{enumerate}
\end{frame}

% ===== SECTION 3: LEGAL FRAMEWORK =====
\section{Legal Framework}

\begin{frame}{Dual-Jurisdiction Model}
    \begin{columns}[T]
        \begin{column}{0.48\textwidth}
            \textbf{Dutch Law (Substantive)}
            \begin{itemize}
                \item Auteurswet 1912 (Copyright Act)
                \item Strong moral rights tradition
                \item EU InfoSoc Directive 2001/29/EC
                \item Article 25: Integrity right
            \end{itemize}

            \vspace{0.3cm}

            \textbf{Advantages:}
            \begin{itemize}
                \item Robust IP protection
                \item AI-forward jurisprudence
                \item EU-wide applicability
            \end{itemize}
        \end{column}

        \begin{column}{0.48\textwidth}
            \textbf{Scottish Courts (Enforcement)}
            \begin{itemize}
                \item Court of Session jurisdiction
                \item Digital rights expertise
                \item Hague Convention (2005)
                \item International enforceability
            \end{itemize}

            \vspace{0.3cm}

            \textbf{Advantages:}
            \begin{itemize}
                \item Predictable outcomes
                \item Cultural sensitivity
                \item Global judgment recognition
            \end{itemize}
        \end{column}
    \end{columns}

    \vspace{0.5cm}

    \textbf{Legal Basis:} Rome I Regulation (EU 593/2008), Hague Convention on Choice of Court Agreements (2005)
\end{frame}

\begin{frame}[fragile]{Core Legal Provisions}
    \begin{block}{Clause 1.2: AI Consent Requirement}
        Use of the Original Work by Non-Interpretive (NI) systems requires \textbf{explicit, prior, written consent} from the Licensor.
    \end{block}

    \textbf{Legal Analysis:}
    \begin{itemize}
        \item Contract law basis (license = contract)
        \item Consent modelled on GDPR Article 6(1)(a)
        \item Breach of contract remedies available
        \item \textit{Burden of proof:} AI developer must prove consent obtained
    \end{itemize}

    \vspace{0.3cm}

    \begin{block}{Clause 2.3: Metadata Preservation}
        All derivatives must preserve Synthetic Lineage Tags (SLTs). Removal = material breach.
    \end{block}

    \textbf{Legal Basis:}
    \begin{itemize}
        \item EU InfoSoc Directive, Article 7 (prohibits RMI removal)
        \item Precedent: \textit{IPC Media Ltd v Highbury-Leisure} [2004]
    \end{itemize}
\end{frame}

% ===== SECTION 4: TECHNICAL ARCHITECTURE =====
\section{Technical Architecture}

\begin{frame}[fragile]{Synthetic Lineage Tags (SLTs)}
    \textbf{Purpose:} Machine-readable provenance metadata embedded in AI outputs

    \begin{lstlisting}[language=json,caption={SLT Example (JSON-LD)}]
{
  "@context": "https://palimpsest-license.org/context/v0.4",
  "@type": "SyntheticLineageTag",
  "sourceWork": {
    "title": "Echoes of Displacement",
    "creator": "Amara Okonkwo",
    "license": "Palimpsest-v0.4"
  },
  "cryptographicProof": {
    "algorithm": "CRYSTALS-Dilithium",
    "signature": "304502210083fda..."
  }
}
    \end{lstlisting}

    \textbf{Innovation:} Quantum-resistant signatures ensure long-term verifiability
\end{frame}

\begin{frame}{AIBDP: AI Boundary Declaration Protocol}
    \textbf{Multi-Layer Defense Architecture:}

    \begin{enumerate}
        \item \textbf{DNS TXT Records:} Domain-wide policy (`\_aibdp.example.com`)
        \item \textbf{.well-known/aibdp.json:} Machine-readable manifest
        \item \textbf{HTTP Headers:} Per-request signals (`GenAI-Consent: deny`)
        \item \textbf{HTML Meta Tags:} Resource-level controls
    \end{enumerate}

    \vspace{0.5cm}

    \textbf{Legal Innovation:} Ignoring AIBDP = \textit{procedural breach} (Clause 3.13)
    \begin{itemize}
        \item No need to prove substantial similarity in AI output
        \item Easier standard of proof than traditional infringement
        \item Analogous to robots.txt but with legal enforceability
    \end{itemize}
\end{frame}

% ===== SECTION 5: EMPIRICAL RESULTS =====
\section{Empirical Results}

\begin{frame}{Case Study: Diaspora Poetry Platform}
    \textbf{Context:} Platform hosting 50,000+ poems from diaspora communities

    \vspace{0.3cm}

    \textbf{Implementation:}
    \begin{itemize}
        \item Phase 1 (1 week): WordPress plugin, AIBDP config
        \item Phase 2 (2 weeks): Bulk SLT generation
        \item Phase 3 (1 week): Consent controls
    \end{itemize}

    \vspace{0.3cm}

    \textbf{Results (6 months):}
    \begin{table}[h]
        \centering
        \begin{tabular}{|l|r|}
            \hline
            \textbf{Metric} & \textbf{Result} \\
            \hline
            Unauthorised scraping reduction & 89\% \\
            \hline
            Consent requests received & 12 \\
            \hline
            Approvals granted (with royalties) & 5 \\
            \hline
            Licensing revenue (6 months) & £15,000 \\
            \hline
            Performance impact & \textless{}2ms latency \\
            \hline
            Creator satisfaction & 94\% \\
            \hline
            Violations enforced & 3 (damages awarded) \\
            \hline
        \end{tabular}
    \end{table}
\end{frame}

% ===== SECTION 6: DISCUSSION =====
\section{Discussion}

\begin{frame}{Contributions to Knowledge}
    \begin{enumerate}
        \item \textbf{Legal Innovation}
            \begin{itemize}
                \item First IP license specifically designed for AI consent
                \item Procedural breach doctrine (easier enforcement)
                \item DAO governance legal recognition
            \end{itemize}

        \item \textbf{Technical Innovation}
            \begin{itemize}
                \item Quantum-resistant attribution (CRYSTALS-Dilithium)
                \item AIBDP: web-wide consent signalling protocol
                \item Interoperable metadata standards
            \end{itemize}

        \item \textbf{Sociocultural Impact}
            \begin{itemize}
                \item Emotional lineage protection framework
                \item Cultural heritage safeguards (UNDRIP alignment)
                \item Creator community empowerment
            \end{itemize}
    \end{enumerate}
\end{frame}

\begin{frame}{Limitations and Future Work}
    \textbf{Limitations:}
    \begin{itemize}
        \item Untested in major litigation (precedent-building ongoing)
        \item Reliance on voluntary AI company compliance
        \item Jurisdictional challenges (US fair use doctrine)
        \item Quantum computing timeline uncertainty
    \end{itemize}

    \vspace{0.5cm}

    \textbf{Future Research Directions:}
    \begin{enumerate}
        \item Comparative enforcement study across jurisdictions
        \item Zero-knowledge proof attribution mechanisms
        \item Economic analysis of licensing markets
        \item Longitudinal impact on creator communities
        \item AI ethics framework integration (EU AI Act)
    \end{enumerate}
\end{frame}

% ===== SECTION 7: CONCLUSION =====
\section{Conclusion}

\begin{frame}{Conclusion}
    \textbf{Summary:}
    \begin{itemize}
        \item Palimpsest License provides legally enforceable, technically robust framework for protecting creative work in AI age
        \item Dual-jurisdiction model maximises international enforceability
        \item Quantum-resistant cryptography ensures long-term viability
        \item Empirical validation demonstrates practical efficacy
    \end{itemize}

    \vspace{0.5cm}

    \textbf{Broader Impact:}
    \begin{itemize}
        \item Shifts power from AI companies to creators
        \item Enables ethical AI development through clear consent
        \item Protects vulnerable communities (diaspora, Indigenous, marginalised)
        \item Contributes to emerging AI governance frameworks
    \end{itemize}

    \vspace{0.5cm}

    \textbf{Call to Action:} Academic community can contribute through:
    \begin{itemize}
        \item Legal scholarship on enforcement mechanisms
        \item Technical research on cryptographic innovations
        \item Sociological studies on creator community impacts
    \end{itemize}
\end{frame}

% References
\begin{frame}[allowframebreaks]{References}
    \printbibliography
\end{frame}

% Acknowledgements
\begin{frame}{Acknowledgements}
    \begin{itemize}
        \item Palimpsest Stewardship Council for governance and oversight
        \item Creator community for invaluable feedback and case studies
        \item Legal experts: [Names] for jurisdictional analysis
        \item Technical contributors: Open-source development community
        \item Funding: [Grant agencies, if applicable]
    \end{itemize}

    \vspace{1cm}

    \begin{center}
        \textbf{Questions?}

        \vspace{0.5cm}

        \texttt{your.email@university.ac.uk}

        \texttt{palimpsest-license.org}
    \end{center}
\end{frame}

\end{document}
